\documentclass[russian,a4paper]{article}
\usepackage[russian]{babel}
\usepackage[utf8]{inputenc}
\usepackage{amsmath}
\usepackage{amsfonts}
\usepackage{color}
\usepackage{verbatim}
\usepackage{graphicx}  

\pagestyle{empty}

\textwidth=17cm
\hoffset=0cm
\voffset=0cm
\headheight=0cm
\topmargin=0cm
\oddsidemargin=0cm

%
% Описание команды для отображения пустой
% цепочки
%
\newcommand{\eps}{\varepsilon}

%
% Команда для отображения 
% нетерминального символа. 
% Он голубеньким подсвечивается
%
\newcommand{\ns}[1]{\textcolor{blue}{<#1>}}
\newcommand{\ls}[1]{\textcolor{red}{<#1>}}

%
% Терминал, он просто чёрный
%
\newcommand{\ts}[1]{<#1>}

\title{Грамматика для модельного языка Python}
\author{}
\date{}

\begin{document}

\maketitle

%
% Вариант 1
%
\section{Введение}

Сперва будет описана контекстно свободная грамматика, которая собственно задаёт синтаксис 
модельного языка Python, а затем к ней будет описан язык задания лексем языка с помощью конечного автомата с действиями.

\section{Описание контекстно свободной грамматики}

Грамматика состоит из следующих нетерминальных символов.
\begin{itemize}
\item \ns{start} - стартовый символ грамматики
\item \ns{single\_input}  - на вход программы поступает файл
\item \ns{file\_input} -- программа запущена в интерактивном режиме
 \item \ns{stmt}  -- оператор языка.
 \item \ns{simple\_stmt}  -- простой оператор, состоящий из малых операторов и перевода строки.
 \item \ns{small\_stmt}  -- состоит из оператора-выражения и других операторов.
 \item \ns{expr\_stmt}  -- оператор - выражение на языке.
 \item \ns{pass\_stmt}  -- оператор пропуска.
 \item \ns{flow\_stmt}  -- оператор управления потоком вычисления.
 \item \ns{break\_stmt}  -- оператор досрочного выхода из цикла.
 \item \ns{continue\_stmt}  -- оператор перехода на следующую итерацию цикла.
 \item \ns{return\_stmt}  -- оператор возврата из функции.
 \item \ns{global\_stmt}  -- оператор обьявляения переменной глобальной.
\\
 \item \ns{compound\_stmt}  -- составной оператор.
 \item \ns{funcdef}  -- описание функии.
 \item \ns{parameters}  -- параметры функции.
 \item \ns{argslist}  -- список аргументов.
 \item \ns{if\_stmt}  -- условный оператор.
 \item \ns{while\_stmt}  -- оператор цикла с предусловие.
 \item \ns{for\_stmt}  -- оператор цикла с итерированием.
 \item \ns{suite}  -- блок кода, относяйщися к составным операторам.
\\ 
 \item \ns{testlist}  -- перечисление выражений.
 \item \ns{exprlist}  -- перечисление арифметических выражений.
 \item \ns{test}  -- единичное выражение .
 \item \ns{or\_test}  -- выражение с или.
 \item \ns{and\_test} --  выражение с и.
 \item \ns{not\_test}  -- выражение с отрицанием.
 \item \ns{comparison}  -- выражение сравнение.
 \item \ns{comp\_op}  -- перечисление лексем, участвующих в сравнении.
 \item \ns{arith\_expr}  -- арифметическое выражение.
 \item \ns{sign}  -- знак.
 \item \ns{term}  -- слагаемое.
 \item \ns{mul\_op}  -- операции с приоритетом у умножения.
 \item \ns{factor}  -- множитель.
 \item \ns{power}  -- степень.
 \item \ns{atom}  -- атом, минимальная частица выражения, состоящая из кортежей, списков, чисел и так далее.

 \item \ns{trailer}  -- постфикс после атома: (), [], . . 
 \item \ns{subscriptlist}  -- список возможных индексов.
 \item \ns{subscript}  -- индекс или секция. 
 \item \ns{sliceop}  -- шаг секции.
\end{itemize}

Список список терминальных символов -- лексем следующий.
\begin{itemize}
\item \ts{SINGLE} -- признак того, что программа запущена в интерактивном режиме.
\item \ts{FILE} -- признак того, что программа запущена в режиме обработки файла
\item \ts{NEWLINE} -- новая строка.
\item \ts{INDENT} -- отступ.
\item \ts{DEDENT} -- конец отступа.
\item \ts{NAME} -- название переменной.
\item \ts{STRING} -- строка в кавычках.
\item \ts{NUMBER} -- число.
\item \ts{COM} -- комментарий, начинающийся с \#.
\item \ts{ENDMARKER} -- признак конца (конец файла).
\item Таблица TW -- служебных слов языка.
\item Таблица TD -- таблица смиволов-разделителей.
\item Таблица TID -- таблица идентификаторов
\end{itemize}

Собственно сама контекстно свободная грамматика.
{\scriptsize
\begin{gather*}
	\ls{start} \to \ts{SINGLE} \ns{single\_input} | \ts{FILE} \ns{file\_input}\\
      \ls{single\_input} \to \ts{NEWLINE} | \ns{simple\_stmt} | \ns{compound\_stmt} \ts{NEWLINE} \\
	\ls{file\_input} \to \{ \ts{NEWLINE} | \ns{stmt} \} \ts{ENDMARKER}\\ 
	\\
	\ls{stmt} \to \ns{simple\_stmt} | \ns{compound\_stmt}\\ 
	\ls{simple\_stmt} \to \ns{small\_stmt} \ts{NEWLINE}\\ 
	\ls{small\_stmt} \to \ns{expr\_stmt}  | \ns{flow\_stmt} | \ns{global\_stmt}|'pass' \\ 
	\ls{expr\_stmt} \to \ns{test} \lbrack '=' (\ns{test}) \rbrack \\ 
	\\
	\ls{flow\_stmt} \to 'break'| 'continue' | 'return' [\ns{testlist}]\\ 
	\ls{global\_stmt} \to 'global' \ts{NAME} \{',' \ts{NAME}\}\\ 
	\ls{compound\_stmt} \to \ns{if\_stmt} | \ns{while\_stmt} | \ns{for\_stmt} | \ns{funcdef}\\ 
	\ls{funcdef} \to 'def' \ts{NAME} \ns{parameters} ':' \ns{suite}\\ 
	\ls{parameters} \to '(' [\ns{argslist}] ')'\\ 
	\ls{argslist} \to \ts{NAME} \{',' \ts{NAME}\}\\ 
	\ls{if\_stmt} \to 'if' \ns{test} ':' \ns{suite} \lbrack 'else'\ ':' \ns{suite} \rbrack \\ 
	\ls{while\_stmt} \to 'while' \ns{test} ':' \ns{suite}\\ 
	\ls{for\_stmt} \to 'for' \ns{exprlist} 'in' \ns{testlist} ':' \ns{suite}\\ 
	\ls{suite} \to \ns{simple\_stmt} |  \ts{INDENT} \ns{stmt} \{\ns{stmt}\} \ts{DEDENT}\\
	\\
	\ls{testlist} \to \ns{test} \{',' \ns{test}\}\\ 
	\ls{exprlist} \to \ns{arith\_expr} \{',' \ns{arith\_expr}\}\\ 
	\ls{test} \to \ns{and\_test} \{'or' \ns{and\_test}\}\\ 
	\ls{and\_test} \to \ns{not\_test} \{'and' \ns{not\_test}\}\\ 
	\ls{not\_test} \to 'not' \ns{not\_test} | \ns{comparison}\\ 
	\ls{comparison} \to \ns{arith\_expr} \{\ns{comp\_op} \ns{arith\_expr}\}\\ 
	\ls{comp\_op} \to '<'|'>'|'=='|'>='|'<='|'!='|'in'|'not'\ 'in'\\ 
	\ls{arith\_expr} \to \ns{term} \{\ns{sign} \ns{term}\}\\ 
	\ls{sign} \to '+'|'-'\\ 
	\ls{term} \to \ns{factor} \{\ns{mul\_op} \ns{factor}\}\\ 
	\ls{mul\_op} \to '*'|'/'|'\%'|'//'\\ 
	\ls{factor} \to [\ns{sign}] \ns{factor} | \ns{power}\\ 
	\ls{power} \to \ns{atom} {\ns{trailer}} ['**' \ns{factor}]\\ 
	\ls{atom} \to'(' [\ns{testlist}] ')' | '[' [\ns{testlist}] ']' | \ts{NAME} | \ts{NUMBER} | \ts{STRING} | 'None' | 'True' | 'False'\\ 
	\\
	\ls{trailer} \to '(' [\ns{testlist}] ')' | '[' \ns{subscriptlist} ']' | '.' \ts{NAME}\\ 
	\ls{subscriptlist} \to \ns{subscript} {',' \ns{subscript}}\\ 
	\ls{subscript} \to \ns{test} | [\ns{test}] ':' [\ns{test}] [\ns{sliceop}]\\ 
	\ls{sliceop} \to ':' [\ns{test}]
\end{gather*}
}

\section{Лексический разбор}

Для лексического разбора опишем диаграмму состояний лексического анализатора модельного языка
А также опишем ранее указанные таблицы TD и TW\\
\begin{table}[h]
\caption{Таблица разделителей -- TD} %title of the table
\centering % centering table
\begin{tabular}{cccccccc} % creating eight columns
\hline\hline %inserting double-line
+&  -&   *&       **&      /& //& \% \\
<& >& <=& >=&  ==& !=& (&\\
)&       [&       ]&     ,&       :&  .& =&\\[1ex]
\hline % inserts single-line
\end{tabular}
\label{tab:hresult}
\end{table}

\begin{table}[h]
\caption{Таблица служебных слов - TW} %title of the table
\centering % centering table
\begin{tabular}{cccccccc} % creating eight columns
\hline\hline %inserting double-line
False&      return&     None& continue& for&  not&\\
True&       def& while& and &       global&  in&\\
if&         or&   else&       pass&       break &    \\ [1ex]
\hline % inserts single-line
\end{tabular}
\label{tab:hresult}
\end{table}

Таблица TID является таблицей с идентфикаторами, которые встретились в программе, она заполняется во время работы анализатора.
\\
\\
На следующей странице собственно представлена таблица состояний. Обозначание '\_' обозначает пробельный символ, '\_ \_ \_ \_' означает 4 пробела. Переход осуществляется по соответствующим символам дугам. Если по текущему символу нельзя перейти ни по одной дуге, то выводится ошибка. Переменная ind изначально инициализирована 0\\
\\
Состояние \ts{SINGLE} или \ts{FILE} не указаны в автомате, одно из них выдается в начале лексического разбора по признаку того, как запщуена программа

\begin{figure} 
  \includegraphics[width=\textwidth, height = 25 cm]{Lex.png}
\end{figure}

\end{document}

